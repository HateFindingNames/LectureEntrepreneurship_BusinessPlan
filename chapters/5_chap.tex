\chapter{Ort, Marketing und Vertrieb}

Für die Anfangszeit operiert das Unternehmen teilweise aus dem HomeOffice und teilweise aus den Räumlichkeiten der Hochschule RheinMain heraus. 
Sobald sich abzeichnet, dass dies nicht mehr ausreichend ist, werden geeignete Büroräumlichkeiten gefunden werden müssen. 

Aktuell ist der Sitz von \textsc{Green Wall} in Mörfelden bei Darmstadt, sobald das Unternehmen über eigene Räumlichkeiten verfügt, wird der Hauptsitz dorthin verlegt werden. 
Aufgrund der zentralen Lage innerhalb Deutschlands wird angestrebt weiterhin im Rhein-Main Gebiet zu bleiben. 

Für den Vertrieb der Module von \textsc{Green Wall} ist für die ersten drei Jahre noch keine eigene Abteilung geplant. 
Dies wird zu späterem Zeitpunkt, wenn die Module in Serienfertigung gehen, neu evaluiert werden müssen. 
Zunächst wird nur der Verkauf innerhalb Deutschlands anvisiert, bei entsprechendem Erfolg der Module und sofern es die Produktionskapazitäten zulassen, könnte noch das nähergelegene europäische Ausland beliefert werden. 

Zu Informations- und Marketingzwecken wird eine Webseite eingerichtet, über die zu späterem Zeitpunkt auch die Vorbestellungen der Module entgegengenommen werden können. 

    \section{Marketing}

        \textsc{Green Wall} setzt beim Marketing auf eine zweigleisige Strategie.
        Im ersten Handlungsbereich sollen Unternehmen, Städte und Gemeinden, sowie andere potenzielle Kunden direkt adressiert werden.
        Hierzu soll auf den relevanten Messen zu den Themen Biomasseproduktion, Einfang und Umwandlung von Kohlendioxid und erneuerbare Energiequellen ein Stand aufgebaut und das Projekt vorgestellt werden.
        Weiterhin sollen Firmen aus Geschäftsfeldern, die für einen hohen Kohlendioxidausstoß bekannt sind, direkt kontaktiert und auf \textsc{Green Wall} aufmerksam gemacht werden.
        Auch Gemeinden, deren Siedlungsgebiet in direkter Nähe zur Autobahn oder zu Zugstrecken liegt sollten angesprochen werden.

        Der zweite Handlungsbereich dient zur Verbesserung der allgemeinen Sichtbarkeit.
        Dabei wird hauptsächlich auf kurze Werbevideos auf YouTube und verwandten Plattformen gesetzt, aber auch Inserate in relevanten Fachzeitschriften kommen infrage.
        Dies zielt in erster Linie darauf ab, die Mitarbeitenden potenzieller Kundenfirmen zu erreichen, welche dann hoffentlich ihre Vorgesetzten auf die Möglichkeiten des CO2-Ausgleichs, welche \textsc{Green Wall} anbieten möchte, aufmerksam machen.
        Weiterhin soll die allgemeine Bevölkerung für die technologischen Möglichkeiten im Kampf gegen die globale Erwärmung sensibilisiert werden.

    \section{Geschäftsform}

    Alle drei GründerInnen sind mit \qty{33.3}{\percent} am Unternehmen beteiligt und stellen jeweils aus privaten Mitteln \qty{10}{\kilo\EUR} 10.000 € für die Gründung von \textsc{Green Wall} zur Verfügung.

    Als Geschäftsform wird die GmbH gewählt, welche mit einem Stammkapital von 30.000 € in das Handelsregister eingetragen werden kann. 
    Der Betrag ist so gewählt, dass das erforderliche Mindestkapital von 25.000 € überschritten wird. 
    Dies dient in erster Linie zur Absicherung des Privatvermögens der GründerInnen, sowie zur Erleichterung der Beteiligung von privaten Investoren. 
    