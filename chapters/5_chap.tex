\chapter{Ort, Marketing und Vertrieb}

Aktuell ist der Sitz von \textsc{Green Wall} in Mörfelden bei Darmstadt, sobald das Unternehmen über eigene Räumlichkeiten verfügt, wird der Hauptsitz dorthin verlegt.
Aufgrund der zentralen Lage innerhalb Deutschlands wird angestrebt, weiterhin im Rhein-Main-Gebiet zu bleiben.
Für den Vertrieb der Module von \textsc{Green Wall} ist für die ersten drei Jahre noch keine eigene Abteilung geplant.
Dies wird zu späterem Zeitpunkt, wenn die Module in Serienfertigung gehen, neu evaluiert.
Zunächst wird nur der Verkauf innerhalb Deutschlands anvisiert.
% Bei entsprechendem Erfolg der Module und sofern es die Produktionskapazitäten zulassen, könnte noch das näher gelegene europäische Ausland beliefert werden.


Zu Informations- und Marketingzwecken wird eine Webseite eingerichtet, über die zu späterem Zeitpunkt auch Vorbestellungen der Module entgegengenommen werden können.

    \section{Marketing}

        \textsc{Green Wall} setzt beim Marketing auf eine zweigleisige Strategie.
        In der unmittelbaren Kundenakquise werden Unternehmen, Städte und Gemeinden, sowie andere potenzielle Kunden direkt adressiert.
        Firmen aus Geschäftsfeldern, die für einen hohen Kohlendioxidausstoß bekannt sind, und Gemeinden, deren Siedlungsgebiet in direkter Nähe zur Autobahn oder zu Zugstrecken liegt, werden direkt kontaktiert um sie von \textsc{Green Wall}s Vorteilen zu überzeugen.
        Ergänzend ist \textsc{Green Wall} auf einschlägigen Fachmessen zugegen.
        Hierzu soll auf den Fachmessen zu den Themen Biomasseproduktion, Einfang und Umwandlung von Kohlendioxid und erneuerbare Energiequellen ein Stand aufgebaut und das Projekt vorgestellt werden.\par\medskip
        %
        Der zweite Handlungsbereich dient zur Verbesserung der allgemeinen Sichtbarkeit.
        Dabei wird hauptsächlich auf kurze Werbevideos auf online Plattformen wie YouTube, aktive Präsenz auf Social Media (Mastodon, X/Twitter/Instagram) gesetzt, aber auch Inserate in relevanten Fachzeitschriften kommen infrage.
        % Dies zielt in erster Linie darauf ab, die Mitarbeitenden potenzieller Kundenfirmen zu erreichen, welche dann hoffentlich ihre Vorgesetzten auf die Möglichkeiten des \(CO_2\)-Ausgleichs, welche \textsc{Green Wall} anbieten möchte, aufmerksam machen.
        % Weiterhin soll die allgemeine Bevölkerung für die technologischen Möglichkeiten im Kampf gegen die globale Erwärmung sensibilisiert werden.

    \section{Geschäftsform}

    Alle drei GründerInnen sind mit \qty{33.3}{\percent} am Unternehmen beteiligt\footnote{Nach Investorenanteilen.} und stellen jeweils aus privaten Mitteln \qty{10}{\kilo\EUR} für die Gründung von \textsc{Green Wall} zur Verfügung.

    Geschäftsform wird eine GmbH sein, welche mit einem Stammkapital von \qty{30}{\kilo\EUR} in das Handelsregister eingetragen werden kann. 
    Der Betrag ist so gewählt, dass das erforderliche Mindestkapital von \qty{25}{\kilo\EUR} überschritten wird. 
    Dies dient in erster Linie zur Absicherung des Privatvermögens der GründerInnen, sowie zur Erleichterung der Beteiligung von privaten Investoren. 
    