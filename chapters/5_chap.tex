\chapter{Ort, Marketing und Vertrieb}

    Für die Anfangszeit operiert das Unternehmen teilweise aus dem HomeOffice und teilweise aus den Räumlichkeiten der Hochschule RheinMain heraus. 
    Sobald sich abzeichnet, dass dies nicht mehr ausreichend ist, werden geeignete Büroräumlichkeiten gefunden werden müssen.

    \section{Marketing}

        \textsc{Green Wall} setzt beim Marketing auf eine zweigleisige Strategie.
        Im ersten Handlungsbereich sollen Unternehmen, Städte und Gemeinden, sowie andere potenzielle Kunden direkt adressiert werden.
        Hierzu soll auf den relevanten Messen zu den Themen Biomasseproduktion, Einfang und Umwandlung von Kohlendioxid und erneuerbare Energiequellen ein Stand aufgebaut und das Projekt vorgestellt werden.
        Weiterhin sollen Firmen aus Geschäftsfeldern, die für einen hohen Kohlendioxidausstoß bekannt sind, direkt kontaktiert und auf \textsc{Green Wall} aufmerksam gemacht werden.
        Auch Gemeinden, deren Siedlungsgebiet in direkter Nähe zur Autobahn oder zu Zugstrecken liegt sollten angesprochen werden.

        Der zweite Handlungsbereich dient zur Verbesserung der allgemeinen Sichtbarkeit.
        Dabei wird hauptsächlich auf kurze Werbevideos auf YouTube und verwandten Plattformen gesetzt, aber auch Inserate in relevanten Fachzeitschriften kommen infrage.
        Dies zielt in erster Linie darauf ab, die Mitarbeitenden potenzieller Kundenfirmen zu erreichen, welche dann hoffentlich ihre Vorgesetzten auf die Möglichkeiten des CO2-Ausgleichs, welche \textsc{Green Wall} anbieten möchte, aufmerksam machen.
        Weiterhin soll die allgemeine Bevölkerung für die technologischen Möglichkeiten im Kampf gegen die globale Erwärmung sensibilisiert werden.

    \section{Geschäftsform}

        Als Geschäftsform wird die GmbH gewählt.
        Die beiden Gründer Julian Uffelmann und Dennis Hunter sind mit jeweils \qty{51}{\percent} und \qty{49}{\percent} am Unternehmen beteiligt. 