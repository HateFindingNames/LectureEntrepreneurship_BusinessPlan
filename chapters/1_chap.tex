\chapter{Zusammenfassung}

Die Konsequenzen des Klimawandels sind nicht mehr zu übersehen und es ist ebenso offensichtlich, dass das Einsparen von CO2 nicht ausreichen wird, um unterhalb der Grenzwerte zu bleiben, welche im Pariser Klimaabkommen festgelegt wurden. 
Neben Aufforstungsprojekten wird auch große Hoffnung in technische Möglichkeiten gesetzt, welche das CO2 wieder aus der Atmosphäre filtern können. 
Es gibt verschiedene Möglichkeiten solche sogenannten ``carbon capture units'' (CCU) zu realisieren.

Green Wall setzt hierbei auf einen bereits etablierten Mechanismus, das Wachstum von Algen. 
Diese werden in dünnen, vertikal aufgestellten Tanks kultiviert. 
Durch das vergleichsweise schnelle Wachstum von Algen kann innerhalb von kurzer Zeit viel Biomasse generiert und auch entsprechend viel CO2 gebunden werden. 
Die entstandene Biomasse kann als Rohstoff für weitere Produkte dienen, beispielsweise Biokraftstoff oder Viehfutter. 
Kunden von Green Wall erhalten durch die Algentanks also nicht nur eine Möglichkeit die eigene CO2-Bilanz zu kompensieren, sondern durch den Absatz der Biomasse auch noch eine weitere Einnahmequelle.
 
Die Aufstellung der Module erfolgt idealerweise so, dass landwirtschaftliche Flächen entlastet werden und Raum für mehr Biodiversität entsteht.
 
Bis die ersten Module vorbestellt werden können, ist jedoch noch Entwicklungsarbeit notwendig, welche innerhalb der nächsten zwei bis drei Jahre geleistet werden soll.
 
Das Geschäftsmodell von Green Wall stützt sich darauf, dass Firmen in Europa per Gesetz dazu verpflichtet sind, innerhalb der nächsten Jahre klimaneutral zu werden. Dies ist jedoch mit einem gewissen Risiko verknüpft, da diese Politik nicht zwangsweise bestehen bleiben muss. 
