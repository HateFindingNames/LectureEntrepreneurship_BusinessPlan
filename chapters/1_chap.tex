\chapter{Zusammenfassung}

Die Konsequenzen des Klimawandels sind nicht mehr zu übersehen und es ist ebenso offensichtlich, dass das Einsparen von \(CO_2\) nicht ausreichen wird, um unterhalb der Grenzwerte zu bleiben, welche im Pariser Klimaabkommen festgelegt wurden. 
Neben Aufforstungsprojekten wird auch große Hoffnung in technische Möglichkeiten gesetzt, welche das \(CO_2\) wieder aus der Atmosphäre filtern können. 
Es gibt verschiedene Möglichkeiten solche ``Carbon Capture and Storage/Utilization'' Einheiten (CCS/U) zu realisieren.\par\medskip
%
\textsc{Green Wall} setzt hierbei auf einen bereits etablierten Mechanismus -- das Wachstum von Algen.
Diese werden in dünnen, vertikal aufgestellten Tanks kultiviert.
Durch das schnelle Algenwachstum kann innerhalb kurzer Zeit vergleichsweise viel Biomasse generiert und entsprechend viel \(CO_2\) gebunden werden.
Die entstandene Biomasse kann als Rohstoff für weitere Produkte beispielsweise Biokraftstoff oder Viehfutter dienen oder, um es der Atmosphäre permanent zu entziehen, eingelagert werden.
Kunden von \textsc{Green Wall} erhalten durch die Algentanks also nicht nur eine Möglichkeit die eigene \(CO_2\)-Bilanz zu kompensieren, sondern durch den Absatz der Biomasse auch noch eine weitere Einnahmequelle.

Die Installation der Module erfolgt so, dass landwirtschaftliche Flächen entlastet werden und Raum für mehr Biodiversität entsteht.

Bis die ersten Module vorbestellt werden können, ist jedoch noch Entwicklungsarbeit notwendig, welche innerhalb der nächsten zwei bis drei Jahre geleistet werden soll.

Das Geschäftsmodell von \textsc{Green Wall} stützt sich darauf, dass Industrie und staatliche Infrastruktur in Europa qua Gesetz dazu verpflichtet sind, bis 2045 de-facto klimaneutral zu werden.
% Dies ist jedoch mit einem gewissen Risiko verknüpft, da diese Politik nicht zwangsweise bestehen bleiben muss.
