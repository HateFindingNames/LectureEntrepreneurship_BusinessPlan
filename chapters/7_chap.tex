\chapter{Risikoanalyse}

    ALGAALGA hat es sich zum Ziel gesetzt eine außerhalb von Laborbedingungen noch sehr unerforschte Technologie am Markt attraktiv zu machen.
    Folgend sollen die größten damit verbundenen Risiken abgeschätzt werden um basierend auf den Erkenntnissen Szenarien skizzieren zu können.
    Aus diesen Überlegungen wird letztlich eine Empfindlichkeitsanalyse abgeleitet -- wie schnell und wie stark reagiert die Unternehmung auf Änderungen äußerer Schlüsselfaktoren?\par\medskip

    Strategien zur Minderung der Schlüsselrisiken werden dargelegt gefolgt von herausgestellten Leistungsindikatoren anhand derer ein fortlaufendes Monitoring der finanziellen und operativen Gesundheit der Unternehmung ermöglicht werden kann.

    \section{Schlüsselrisiken}

        Kern-Risikofaktoren sollen identifiziert werden um ihre Einflüsse auf die verschiedenen Aspekte der Planungs- Gründungs- und Etablierungsphase zu beurteilen.

        \subsection{Markt}

            Mit dem Krieg in der Ukraine haben sich Energiepreise in allen Bereichen deutlich erhöht.
            Der Energiepreismarkt hat sich jüngst zwar wieder beruhigt und vor allem Gaspreise sind wieder 

        \subsection{Regulatorisch}
            Da ALGAALGA einen sich in der Entwicklung befindlichen, durch staatliche Regularien künstlich geschaffen Markt bedienen will, ist in der Volatilität der politischen Situation der größte Risikofaktor für Akzeptanz und Absatz zu erwarten.
            ALGAALGAs Produkte sind umso attraktiver, je intensiver klimafreundliche Ökonomien vorangetrieben werden. 
            Wissenschaftlich quasi-sicher ist die unbedingte Notwendigkeit einer globalen Emissionsbilanz von netto-null.
            Die klimapolitische Weltlage ist in den Industrienationen insgesamt als volatil zu bezeichnen.
            Während sich innerhalb der EU durchaus der Wille zu signifikant reduzierten CO2-Emissionen abzeichnet, ist unklar, wie sich der Ausgang der kommenden US-amerikanischen Wahl auf die politischen Mehrheitsverhältnisse in Europa und damit letztlich auf klimafreundliche Technologien und Regularien auswirkt.
            
        \subsection{Technologisch}

            Primäre Performance-Metrik ist die Menge absorbierten und damit abgeschiedenen Kohlenstoffdioxids je Einheit Standfläche, Kosten und eingetragener Energie.
            Sekundäre Metriken sind Wartungsaufwand und Fehleranfälligkeit.
            \begin{itemize}
                \item Effizienz und Ertrag: die Menge anfallender Biomasse korreliert unmittelbar mit der Menge abgeschiedenen Kohlenstoffdioxids.
                Abhängig von den Umgebungsbedingungen kann es zu starken Fluktuationen des Ertrages kommen.
                Da die Systeme ohne künstliche Beleuchtung und externe  auskommen müssen, sind genetisch modifizierte Algen 
                \item Anfälligkeit für Kontamination: das aktive Medium ist ein lebender, pflanzlicher Organismus. Wie alle lebenden Organismen ist er anfällig für chemische, biologische und radiative Belastungen.
                \item Technologischer Fortschritt: im umkämpften Markt kann rasche Weiterentwicklungen der Technologie das eigene System obsolet machen.
            \end{itemize}
        \subsection{Finanzierung}

        \subsection{Operativ}
            ALGAALGAs in bestehende industrielle Prozesse und Strukturen integrierbaren Systeme erfordern gut ausgebildetes Personal.
            Da vor allem in der Markteinführungsphase nicht mit der notwendigen Liquidität zu erwarten, höhere Gehälter zahlen zu können, muss von dem Personal über ihre Fachexpertise hinaus ein erhöhtes Maß an Motivation und Loyalität erwartet werden.\par\medskip

            Regelmäßiger Umsatz ist durch Wartungsverträge sichergestellt.
            Dies setzt Personal in geeigneter Zahl und mit nötiger Expertise voraus.
    
    \section{Szenarien}

        \subsection{Best Case}

        \subsection{Worst Case}

        \subsection{Most-likely Case}

        \subsection{Empfindlichkeitsanalyse}

    \section{Risikominderungsstrategien}

        \subsection{Diversifikation}

        \subsection{Hedging?}
        % wahrscheinlich overkill. das alles hier ist wahrscheinlich overkill...

    \section{Leistungsindikatoren}
    % Leistungsindikatoren identifizieren um standardisiert die gesundheit des unternehmens zu beobachten