\chapter{Risikoanalyse}

\textsc{Green Wall} hat es sich zum Ziel gesetzt eine außerhalb von Laborbedingungen noch sehr unerforschte Technologie am Markt attraktiv zu machen.
Folgend sollen die größten damit verbundenen Risiken abgeschätzt werden, um basierend auf den Erkenntnissen Szenarien skizzieren zu können.
Aus diesen Überlegungen wird letztlich eine Empfindlichkeitsanalyse abgeleitet -- wie schnell und wie stark reagiert die Unternehmung auf Änderungen äußerer Schlüsselfaktoren?\par\medskip

Strategien zur Minderung der Schlüsselrisiken werden dargelegt, gefolgt von herausgestellten Leistungsindikatoren, anhand derer ein fortlaufendes Monitoring der finanziellen und operativen Gesundheit der Unternehmung möglich gemacht wird.

\section{Schlüsselrisiken}

Kern-Risikofaktoren sollen identifiziert werden, um ihre Einflüsse auf die verschiedenen Aspekte der Planungs-, Gründungs- und Etablierungsphase zu beurteilen.

\subsection{Markt}

Mit dem Krieg in der Ukraine haben sich Energiepreise in allen Bereichen deutlich erhöht.
Der Energiepreismarkt beruhigte sich jüngst zwar wieder und vor allem die Belastung durch hohe Gaspreise sank, beide liegen jedoch Europaweit noch immer deutlich über Vorkriegsniveau \cref{fig:energiepreise}.

\begin{figure}[h]
    \centering
    \includesvg[width=\textwidth]{stat/EnergyPriceDev/eg_prices_eu}
    \caption[Energiepreisentwicklung ausgewählter Länder in der EU]{Energiepreisentwicklung ausgewählter Länder in der EU \cite{Dataset.Eurostat.EnergyStatisticsNaturalGasAndElectricityPricesfrom2007Onwards.2024}.}\label{fig:energiepreise}
\end{figure}

% Gleichzeitig ist

\subsection{Regulatorisch}

Da \textsc{Green Wall} einen sich in der Entwicklung befindlichen, durch staatliche Regularien künstlich geschaffenen Markt bedienen will, ist in der Volatilität der politischen Situation der größte Risikofaktor für Akzeptanz und Absatz zu erwarten.
\textsc{Green Wall}s Produkte sind umso attraktiver, je intensiver klimafreundliche Ökonomien vorangetrieben werden.
Wissenschaftlich quasi-sicher ist die unbedingte Notwendigkeit einer globalen Emissionsbilanz von netto-null.
Die klimapolitische Weltlage ist in den Industrienationen insgesamt als volatil zu bezeichnen.
Während sich innerhalb der EU durchaus der Wille zu signifikant reduzierten \(CO_2\)-Emissionen abzeichnet, ist unklar, wie sich der Ausgang der kommenden US-amerikanischen Wahl auf die politischen Mehrheitsverhältnisse in Europa und damit letztlich auf klimafreundliche Technologien und Regularien auswirkt.

\subsection{Technologisch}

Primäre Performance-Metrik ist die Menge absorbierten und damit abgeschiedenen Kohlenstoffdioxids je Einheit Standfläche, Kosten und eingetragener Energie.
Sekundäre Metriken sind Wartungsaufwand und Fehleranfälligkeit.
\begin{itemize}
    \item Effizienz und Ertrag: die Menge anfallender Biomasse korreliert unmittelbar mit der Menge abgeschiedenen Kohlenstoffdioxids.
          Abhängig von den Umgebungsbedingungen kann es zu starken Fluktuationen des Ertrages kommen.
          Da die Systeme ohne künstliche Beleuchtung auskommen müssen, kommen mittelfristig, hinsichtlich Robustheit und Ertrag, genetische modifizierte Algen zum Einsatz.
    \item Anfälligkeit für Kontamination: das aktive Medium ist ein lebender, pflanzlicher Organismus. Wie alle lebenden Organismen ist er anfällig für chemische, biologische und Strahlenbelastungen.
    \item Technologischer Fortschritt: im umkämpften Markt kann rasche Weiterentwicklungen der Technologie das eigene System obsolet machen.
\end{itemize}

\subsection{Operativ}

Mit einem Großteil der geplanten, regelmäßigen Einnahmen durch Wartungsverträge ist \textsc{Green Wall} stark abhängig von geschultem Personal.
Kontakt zu den Kunden und reibungslose Abläufe tragen zur Kundenbindung bei.
Daher liegt diese Abhängigkeit in verstärkter Form während der Etablierungsphase vor.\par


\subsection{Finanzierung}

Stand September 2022 beträgt die sich in Entwicklung befindliche Kapazität zur \(CO_2\)-Abscheidung weltweit \num{244} Millionen Tonnen \(CO_2\) pro Jahr (\unit{\mega\tonne\per\an}).
Mit einer Kapazität im Vorjahr von rund \qty{170}{\mega\tonne\per\an} verzeichnete sie einen Anstieg von \qty{44}{\percent}.
Global betrugen 2022 die installierten Kapazitäten rund \qty{42.5}{\mega\tonne\per\an}.
Den größten nationalen Marktanteil Europas hat hier Norwegen mit lediglich \qty{4}{\percent}.
Vor dem Hintergrund der voranschreitenden Klimakatastrophe und der Projektion des IPCC sind die geplanten Kapazitäten viel zu gering -- es ist also ein weitaus größerer Zuwachs der installierten Kapazitäten zu erwartet \cite{Book.EJR.CARBONCAPTUREUTILISATIONANDSTORAGEINTHEEUROPEANUNION.2023}.\par\medskip
%
\begin{figure}[h]
    \centering
    \includesvg[width=\textwidth]{stat/GlobVCInCCUS/ccus-startup-vc-investment}
    \caption[Globale Risikokapitalfinanzierung von 2018 bis 2023]{Globale Risikokapitalfinanzierung von 2018 bis 2023 \cite{Statista2022.GlobalVentureCapital}.}\label{fig:glob ccus vc inv}
\end{figure}

\textsc{Green Wall}s in bestehende industrielle Prozesse und Strukturen integrierbaren Systeme erfordern gut ausgebildetes Personal.
Da vor allem in der Markteinführungsphase die notwendige Liquidität nicht vorhanden ist, um höhere Gehälter zahlen zu können, muss von dem Personal über ihre Fachexpertise hinaus ein erhöhtes Maß an Motivation und Loyalität erwartet werden.\par\medskip

Regelmäßiger Umsatz ist durch Wartungsverträge sichergestellt.
Dies setzt Personal in geeigneter Zahl und mit nötiger Expertise voraus.

\section{Szenarien}

\subsection{Best Case}

USA, Kanada und die EU einigen sich auf eine gemeinsame, geplante und kommunizierte \(CO_2\)-Bepreisung.
Industrien und Investoren haben Planungssicherheit und investieren verstärkt in CCS/U-Systeme.
\(CO_2\)-Steuern werden erhoben und an die Bevölkerung in einer Art zurück verteilt, dass ein breites Bewusstsein und insbesondere Akzeptanz zur Minderung der Nettoemissionen entsteht.
\textsc{Green Wall} böte eine gesellschaftlich gewollte und ökonomisch legitimierte Lösung.

\subsection{Worst Case}

Europäisch und global erlangen deutlich mehr Konservative politische Macht.
Infolge ist damit zu rechnen, dass klimaschädliche Subventionen steigen während staatliche Subventionen und Investition in CCS/U stark reduziert bis ganz gestrichen werden.
Der künstlich geschaffene Markt durch \(CO_2\)-Bepreisung würde vergleichsweise abrupt zusammenbrechen.
Auch wenn sich durch Folgeregierungen die Situation wieder entspannen würde, wäre für Zeiträume von Jahren \textsc{Green Wall} die Rentabilität entzogen.

\subsection{Most-likely Case}

Die beteiligten Nationen -- mit möglichen kleinen Verschiebungen um wenige Jahre -- werden sich an die vereinbarten Klimaziele halten.
Dies erscheint insbesondere wahrscheinlich, als das jüngst die folgen klimaschädlichen Wirtschaftens den Regierungen und Gesellschaften augenscheinlich wurden.
Die eigenen Projektionen bleiben mittelfristig valide oder werden übertroffen.

\subsection{Empfindlichkeitsanalyse}

Eine globale Minderung der \(CO_2\)-Emissionen auf netto Null ist unumgänglich.
Regulatorische Maßnahmen, die dieser Tatsache keine Rechnung tragen, scheinen am gefährlichsten, da \textsc{Green Wall} hier kaum bis keine Einflussmöglichkeiten hat und gleichzeitig am schwersten getroffen wäre.\par\medskip
%


\section{Risikominderungsstrategien}

\textsc{Green Wall} wird bei seinem Marketing verstärkt auf Imageaufbau achten.
Eine Vermarktung \textit{auch} als ``Live-Style'' Produkt und als Mittel, seinen Kunden beim Aufbau eines grünen bzw. klimafreundlichen Images könnte die schlimmsten Auswirkungen volatiler regulatorischer Landschaften abfedern.
Weiter ist anzunehmen, dass eine frühe Erschließung innereuropäischer Märkte außerhalb des Heimatlandes das Geschäftsmodell durch Risikodiversifikation weniger empfindlich macht.
% wahrscheinlich overkill. das alles hier ist wahrscheinlich overkill...

\section{Leistungsindikatoren}
% Leistungsindikatoren identifizieren um standardisiert die gesundheit des unternehmens zu beobachten

Ein fortlaufendes Monitoring der Gesundheit des Unternehmens soll anhand folgender Indikatoren sichergestellt sein:

\begin{itemize}
    \item Fluktuationsrate des Personals: nicht zu verwechseln mit der Zuwachsrate soll die Rate der Kündigungen vs. Neueinstellungen Aufschluss über die Zufriedenheit des eigenen Personals geben.
    \item Jährlich installierte \(CO_2\)-Bindungskapazität.
    \item Geografische und sektorbezogene Diversität der Kunden.
\end{itemize}

Dies soll nicht als erschöpfend verstanden sein -- mit der Entwicklung der Unternehmung kommen weitere Leistungsindikatoren hinzu.