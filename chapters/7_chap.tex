\chapter{Risikoanalyse}

    \section{Schlüsselrisiken}

        \subsection{Markt}

        \subsection{Regulatorisch}
            Da ALGAALGA einen sich in der Entwicklung befindlichen Markt bedienen will, der durch staatliche Regularien künstlich geschaffen wurde ist hier der größte Risikofaktor zu erwarten.
            ALGAALGAs Produkte sind umso attraktiver, je intensiver klimafreundliche Ökonomien voran getrieben werden. 
            Wissenschaftlich quasi-sicher ist die unbedingte Notwendigkeit einer globalen Emissionsbilanz von netto-null.
            Die klimapolitische Weltlage ist in den Industrienationen insgesamt als volatil zu bezeichnen.
            Während sich innerhalb der EU durchaus der Wille zu signifikant reduzierten CO2-Emissionen abzeichnet ist unklar, wie sich der Ausgang der kommenden US-amerikanischen Wahl auf die politischen Mehrheitsverhältnisse in Europa und damit letztlich auf klimafreundliche Technologien und Regularien auswirkt.
            
        \subsection{Technologisch}

        \subsection{Finanzierung}

        \subsection{Operativ}
    
    \section{Szenarien}

        \subsection{Best Case}

        \subsection{Worst Case}

        \subsection{Most-likely Case}

        \subsection{Empfindlichkeitsanalyse}

    \section{Risikominderungsstrategien}

        \subsection{Diversifikation}

        \subsection{Hedging?}
        % wahrscheinlich overkill. das alles hier ist wahrscheinlich overkill...

    \section{Leistungsindikatoren}
    % Leistungsindikatoren identifizieren um standardisiert die gesundheit des unternehmens zu beobachten