\chapter{Projektidee}

\section{Hintergrund}

Die Projektidee entstand bei der Evaluierung des European Green Deal.
Dort ist festgeschrieben, dass wieder mehr Landwirtschaft und Produktion von Lebensmitteln in Europa stattfinden soll.
Gleichzeitig soll jedoch auch die Fläche von Naturschutzgebieten und Wäldern vergrößert werden.
Bei der Betrachtung der Aufschlüsselung der Flächennutzung innerhalb Europas fällt auf, dass kaum ungenutzten Flächen übrig sind auf denen beide Ziele realisiert werden könnten.
Da es also an Platz fehlt ist der logische Schluss, die vorhandene Fläche effizienter zu nutzen, sowie Flächen zu erschließen, die bisher nicht nutzbar waren.
Diese bisher ungenutzten Flächen werden als Brachland aufgeführt und machen ungefähr \qty{6}{\percent} der Fläche von Europa aus.
Beispielsweise fallen Seitenstreifen von Autobahnen und Bahngleisen in diese Kategorie \cite{Legislation2019.DerEuropaeischeGrueneDeal,Legislation2020.EUBiodiversitaetsstrategieFuer2030,Online.DieGemeinsameAgrarpolitikAufEinenBlick}.\par\medskip
%
Die Idee ist nun, diese oft schmalen Streifen landwirtschaftlich zu nutzen durch die Installation von vertikalen Algentanks, welche ähnlich wie oder anstelle von Lärmschutzwänden aufgebaut werden können.
In ihnen kann durch die Kultivierung von Algen Biomasse produziert werden.
Dabei ist die Wachstumsrate von Algen um ein Vielfaches größer als die von Pflanzen wie Mais oder Weizen auf vergleichbarer Fläche.
Die durch die Algen produzierte Biomasse ist jedoch ebenfalls als Futtermittel für Tiere geeignet.
So können umliegende landwirtschaftliche Flächen entlastet werden und es entsteht Raum für mehr Biodiversität und Naturschutzgebiete.
Zusätzlich würde Europa unabhängiger werden vom Import von Soja als Futtermittel.
Je nachdem welche Algenart verwendet würde und wie viel Biomasse produziert wird, wäre es ebenfalls denkbar diese zu Biodiesel, Nahrungsergänzungsmitteln, sowie Medizin oder Kosmetikprodukten weiterzuverarbeiten.\par\medskip

Die Algentanks von \textsc{Green Wall} sind dabei nicht auf die bisher genannten Aufstellungsorte begrenzt.
Durch die schlanke und modulare Bauweise kommt auch eine Platzierung in Stadtgebieten, beispielsweise als Rückwand von Bushaltestellen und Sitzgelegenheiten, infrage.
Weiterhin könnten die Module als Raumtrenner auf großen Parkplätzen oder generell auf Firmengeländen aufgebaut werden.\par\medskip

Dabei helfen die Algentanks auch bei der Verbesserung der Luftqualität, denn die Algen wandeln während ihres Wachstums Kohlendioxid in Sauerstoff um.
Das aus der Luft gefilterte \(CO_2\) wird dabei in der Biomasse gespeichert.
Bei geeigneter Weiterverarbeitung entsteht so eine negative \(CO_2\)-Bilanz, welche Firmen helfen kann ihre Emissionen auszugleichen.

\section{Das Projekt}

\textsc{Green Wall} hat es sich zur Aufgabe gemacht, europäische Kommunen und Länder bei ihren Bemühungen um die Erreichung und Umsetzung von Klimazielen zu unterstützen.
Wir haben eine doppelte Aufgabe: Wir bieten innovative Lösungen für die anstehenden Herausforderungen im Zusammenhang mit dem Klimawandel und helfen unseren Kunden, ihre Kohlendioxid-Emissionen zu reduzieren.
Durch den Einsatz fortschrittlicher Technologien in der Biomasseproduktion will \textsc{Green Wall} einen wichtigen Beitrag zum weltweiten Kampf gegen den Klimawandel leisten.

\section{Leistungsversprechen}
Die Systeme von \textsc{Green Wall} ermöglichen es die Netto-Kohlendioxidemissionen seiner Kunden erheblich zu reduzieren.
Die Reduktion der Netto-Emissionen hilft nicht nur bei der Einhaltung gesetzlicher Vorschriften, sondern senkt auch die mit dem Kauf von Emissionszertifikaten verbundenen Kosten.
Unsere Technologie bietet im Vergleich zu herkömmlichen Methoden eine um Größenordnungen effizientere Biomasseproduktion pro Flächeneinheit, wodurch die Ressourcennutzung maximiert und der ökologische Fußabdruck minimiert wird.

\section{Produkt Portfolio}

\begin{enumerate}
    \item Einheiten zur Kohlenstoffabscheidung und -verwertung:
          \begin{itemize}
              \item Modulare Mikroalgen-Bioreaktoren: Unser Hauptproduktangebot besteht aus anwendungsspezifischen und optimierten CCS/U-Einheiten in Form von modularen Mikroalgen-Bioreaktoren.
                    Diese Bioreaktoren sind darauf ausgelegt, Kohlendioxid effizient abzufangen und in wertvolle Biomasse umzuwandeln.
              \item Kundenspezifische Anpassung: Jeder Bioreaktor kann auf die spezifischen Bedürfnisse des Kunden zugeschnitten werden, um eine optimale Leistung und Integration in bestehende Systeme zu gewährleisten.
          \end{itemize}
    \item Wartungsverträge:
          \begin{itemize}
              \item Betriebliche Unterstützung: Wir bieten umfassende Wartungsverträge an, um den kontinuierlichen und effizienten Betrieb unserer Bioreaktoren zu gewährleisten.
                    Diese Verträge umfassen regelmäßige Wartung, Fehlersuche und Leistungsoptimierung.
              \item Langfristige Partnerschaft: Durch die kontinuierliche Unterstützung wollen wir langfristige Partnerschaften mit unseren Kunden aufbauen, um einen nachhaltigen Nutzen und hervorragende Betriebsleistungen zu gewährleisten.
          \end{itemize}
    \item Biomasse Ernte und Verkauf:
          \begin{itemize}
              \item Dienstleistungsmodell: Neben dem Verkauf von Bioreaktoren bietet \textsc{Green Wall} ein Dienstleistungsmodell an, das die Ernte und den Verkauf der produzierten Biomasse umfasst.
                    Diese Biomasse kann von Drittunternehmen zur kohlenstoffneutralen Kraftstoffproduktion oder als Viehfutter genutzt werden.
              \item Wirtschaftlicher Nutzen: Dieses Servicemodell bietet nicht nur eine zusätzliche Einnahmequelle für unsere Kunden, sondern trägt auch zur Kreislaufwirtschaft bei, indem das abgeschiedene Kohlendioxid in wertvolle Produkte umgewandelt wird.
          \end{itemize}
\end{enumerate}

\section{Projektkontext}

Die Folgen des Klimawandels lassen sich nicht mehr ignorieren und sind weltweit zu spüren. Mit dem Pariser Klimaabkommen sollte die globale Erderwärmung auf \qty{1,5}{\kelvin} gegenüber der vorindustriellen Zeit begrenzt werden. Inzwischen ist jedoch klar, dass dieses Ziel kaum noch erreicht werden kann und die Zielsetzung wurde auf unter \qty{2}{\kelvin} erhöht. Dazu ist es dringend notwendig, die Emission von Treibhausgasen zu reduzieren, insbesondere den Ausstoß von Kohlendioxid \cite{Online2024.UebereinkommenVonParis}.

Auch wenn es optimal wäre, den Ausstoß von \(CO_2\) gänzlich zu vermeiden, um so Klimaneutralität zu erreichen, ist dies für die meisten Industriesektoren und im notwendigen Zeitraum nicht realistisch.
Daher wird als Gegenmaßnahme für vergangene und aktuelle Emissionen auch darauf gesetzt \(CO_2\) aus der Atmosphäre wieder zu binden.
Mögliche Maßnahmen sind die Aufforstung von Wäldern oder die Bindung von \(CO_2\) durch gezielte Verwitterung von bestimmten Gesteinssorten auf Ackerflächen.
Eine weitere Möglichkeit, auf die viel Hoffnung gesetzt wird, ist, das Kohlendioxid mithilfe von innovativen Technologien sprichwörtlich aus der Luft herauszufiltern.
Bei diesem \textit{direct air capture} genannten Verfahren können verschiedene Varianten zum Einsatz kommen.
Eine davon trägt den Namen \textit{Mammoth}\footnote{https://climeworks.com/plant-mammoth} und wird derzeit in Island erfolgreich implementiert.
Mit der dortigen Anlage sollen jährlich bis zu 36 tausend Tonnen (\unit{\kilo\tonne}) \(CO_2\) aus der Atmosphäre entfernt und gespeichert werden können.
\textit{Mammoth} hat allerdings einen enormen Stromverbrauch und kann nur aufgrund von Islands geothermaler Aktivität rentabel betrieben werden.

\textsc{Green Wall} hingegen verfolgt einen anderen, dezentralen Ansatz: auch hier wird \(CO_2\) der Atmosphäre entzogen, doch der Umweg über das Algenwachstum und die Speicherung in Form von Biomasse benötigen nur einen Bruchteil der elektrischen Energie.
Bei dieser Methode ist es eher wichtig, dass ein möglichst großer Anteil der Algen mit Sonnenlicht bestrahlt wird, was durch die flache Bauweise der Tanks erreicht werden soll.
Dass der Betrieb von Algentanks zur Verbesserung der Luftqualität in Städten, sowie der Anbau und Vertrieb von Algen als Nahrungsergänzungsmittel in Deutschland funktioniert, haben andere Unternehmen wie \textit{LIQUID3}\footnote{https://liquid3.rs/ }, \textit{Heidelberger Chlorella}\footnote{https://www.heidelberger-chlorella.de/ueber-uns/} und \textit{Algenland GmbH}\footnote{https://www.algenland.de/} bereits erfolgreich demonstriert.
Daher ist \textsc{Green Wall} zuversichtlich, dass im großflächigen Anbau von Algen und der Produktion von Biomasse als Kompensation von \(CO_2\) Emissionen noch ein enormes Wachstumspotenzial enthalten ist. 
